\begin{abstract}
Query optimization is the part of the query process in which the database system compares different query strategies and chooses the one with the least expected cost. The query optimizer, which carries out this function, is a key part of the relational database and determines the most efficient way to access data. It makes it possible for the user to request the data without specifying how these data should be retrieved.

The cost of accessing a query is a weighted combination of the I/O and processing costs. The I/O cost is the cost of accessing index and data pages from disk. Processing cost is estimated by assigning an instruction count to each step in computing the result of the query.

In this project, we study the optimization done by PostgreSQL and SQLite by running the queries present in the SSB and TPC-H benchmarks. The results obtained help us to further investigate the query optimization techniques implemented by the two database systems.
\end{abstract}
