\documentclass[sigplan,10pt,nonacm]{acmart}
\settopmatter{printfolios}

\makeatletter
\def\ps@headings{%
\def\@oddhead{\mbox{}\scriptsize\rightmark \hfil \thepage}%
\def\@evenhead{\scriptsize\thepage \hfil \leftmark\mbox{}}%
\def\@oddfoot{}%
\def\@evenfoot{}}
\makeatother
\raggedbottom
\graphicspath{ {./graphs/} }

\pagestyle{plain}
\settopmatter{printfolios=true}

%\usepackage[small,compact]{titlesec}
\usepackage{epsfig}
\usepackage{graphicx}
\usepackage{xspace}
\usepackage{listings}
\usepackage{multirow}
\usepackage{listings}
\usepackage{color}
\usepackage{caption}
\usepackage{subcaption}

\definecolor{mygreen}{rgb}{0,0.6,0}
\definecolor{mygray}{rgb}{0.5,0.5,0.5}
\definecolor{mymauve}{rgb}{0.58,0,0.82}

\lstset{ 
  backgroundcolor=\color{white},   % choose the background color; you must add \usepackage{color} or \usepackage{xcolor}; should come as last argument
  basicstyle=\footnotesize,        % the size of the fonts that are used for the code
  breakatwhitespace=false,         % sets if automatic breaks should only happen at whitespace
  breaklines=true,                 % sets automatic line breaking
  captionpos=b,                    % sets the caption-position to bottom
  commentstyle=\color{mygreen},    % comment style
  deletekeywords={...},            % if you want to delete keywords from the given language
  escapeinside={\%*}{*)},          % if you want to add LaTeX within your code
  extendedchars=true,              % lets you use non-ASCII characters; for 8-bits encodings only, does not work with UTF-8
  %firstnumber=1000,                % start line enumeration with line 1000
  frame=single,	                   % adds a frame around the code
  keepspaces=true,                 % keeps spaces in text, useful for keeping indentation of code (possibly needs columns=flexible)
  keywordstyle=\color{blue},       % keyword style
  language=C,                 % the language of the code
  morekeywords={*,...},            % if you want to add more keywords to the set
  numbers=right,                    % where to put the line-numbers; possible values are (none, left, right)
  numbersep=5pt,                   % how far the line-numbers are from the code
  numberstyle=\tiny\color{black}, % the style that is used for the line-numbers
  rulecolor=\color{black},         % if not set, the frame-color may be changed on line-breaks within not-black text (e.g. comments (green here))
  showspaces=false,                % show spaces everywhere adding particular underscores; it overrides 'showstringspaces'
  showstringspaces=false,          % underline spaces within strings only
  showtabs=false,                  % show tabs within strings adding particular underscores
  %stepnumber=2,                    % the step between two line-numbers. If it's 1, each line will be numbered
  stringstyle=\color{mymauve},     % string literal style
  tabsize=2,	                   % sets default tabsize to 2 spaces
  title=\lstname                   % show the filename of files included with \lstinputlisting; also try caption instead of title
}

\usepackage{url}
%\usepackage[hyphens]{url}
\usepackage{hyperref}
%\hypersetup{breaklinks=true}
%\usepackage[margin=10pt]{subcaption}
%\usepackage{subcaption}

% Other useful macros
\newcommand{\fixme}[1]{\textbf{FIX: #1}}
\newcommand{\codesm}[1]{\texttt{\small #1}}
\newcommand{\implication}{\vspace{0.1in} \noindent \emph{Implication:~}}
\newcommand{\fig}[4]{%
  \begin{figure}[t]%
    %\frame{\includegraphics[#1]{#2}}
    \includegraphics[#1]{#2}
    \caption{{#3}}\label{#4}
\end{figure}
}

\newcommand{\question}[1]{\textbf{DOUBT: #1}}

\def \sec {\S}



\begin{document}
\widowpenalty=5000
\clubpenalty=5000


\title{\textsf{A Study on Query Optimization}}


\author{Anjali(anjali@wisc.edu) and Sakshi(sbansal8@wisc.edu)}


\begin{abstract}
Query optimization is the part of the query process in which the database system compares different query strategies and chooses the one with the least expected cost. The query optimizer, which carries out this function, is a key part of the relational database and determines the most efficient way to access data. It makes it possible for the user to request the data without specifying how these data should be retrieved.

The cost of accessing a query is a weighted combination of the I/O and processing costs. The I/O cost is the cost of accessing index and data pages from disk. Processing cost is estimated by assigning an instruction count to each step in computing the result of the query.

In this project, we study the optimization done by PostgreSQL and SQLite by running the queries present in the SSB and TPC-H benchmarks. The results obtained help us to further investigate the query optimization techniques implemented by the two database systems.
\end{abstract}


\maketitle



\section{Introduction}
\label{sec:intro}

\fig{width=\columnwidth}{query}{\textmd{Stages of Query processing}}{fig:query}

Query optimization is part query processing in many relational database management systems. It determines the most efficient way to execute a particular query by evaluating the various query plans for it. Once the query is submitted to the database server, it is then passed to the parser and then passed to the query optimizer as shown in Figure~\ref{fig:query}.

There are many ways to implement a particular query depending on the schema and the complexity of the query. Different databases use different query optimizer which might result in different execution time for the same query when executed on different platforms. The fundamental task of a query optimizer is to select an algorithm from among the many available options that provides the answer with a minimum of disk I/O and CPU time.

Query optimizer~\cite{ref:sqlite3} frees the programmer from the task of selecting a particular query plan and allows the programmer to focus on high-level application issues. For simple query the choice of plan is mostly obvious but as schema and queries become important, the optimizer plays an inmportant role in simplifying the work of application development for the programmer.

The rest of the paper is organised as follows: Section~\ref{sec:db} gives an overview of the architecture for PostgreSQL and SQLite, Section~\ref{sec:bench} defines the benchmarks in detail, in Section~\ref{sec:results} we discuss the results. Finally, we discuss future work and conclude in Section~\ref{sec:future} and Section~\ref{sec:conclusion} respectively. 



\section{Databases}
\label{sec:db}

\subsection{PostgreSQL}
\fig{width=\columnwidth}{postgres}{\textmd{PostgreSQL Architecture}}{fig:postgres}

PostgreSQL~\cite{ref:pg1} is a relational database management system following a client-server architecture. At the server side an instance is created comprising of PostgreSQL's processes and shared memory which handles the access to the data. Client programs connect to the instance sending read and write operations. The overview of the architecture is shown in Figure~\ref{fig:postgres}.

An instance consists of multiple processes,  postmaster process, multiple postgres processes (one for each connection), WAL writer process, background writer process, checkpointer process and other optional processes like autovacuum launcher process, logger process , archiver process, stats collector process, WAL sender process (when streaming replication is active), WAL receiver process (when streaming replication is active) and background worker processes (in case a query gets parallelized). The client program sends the request to the instance. The instance itself does not directly write to disk instead it buffers the requested data in shared buffer. The flushing to disk is done at a later stage.

\subsection{SQLite}

\fig{width=\columnwidth}{sqlite}{\textmd{SQLite Architecture}}{fig:sqlite}

SQLite~\cite{ref:sqlite2} is an embedded file based relational database management system which can be linked statically or dynamically with the application program. It does not have a client-server database engine which makes the SQLite applications require less configuration. SQLite is called zero-conf as it does not require service management or access control based on password and GRANT.


SQLite complies with ACID  (atomicity, consistency, isolation, durability) properties and implements the SQL standard. It stores the entire database as a single disk file and all reads and write takes place directly on this file. 

SQLite database architecture is divided into two different sections called core and backend. Core contains Interface, Tokenizer, Parser, Code generator, and the virtual machine, which create an execution order for database transactions. Backend contains B-tree, Pager and OS interface to access the file system. Tokenizer, Parser and code generator is together named as the compiler which generates a set of opcodes that runs on a virtual machine.

\section{Benchmarks}
\label{sec:bench}

\subsection{TPC-H}
\fig{width=\columnwidth}{tpch}{\textmd{TPC-H schema}}{fig:tpch}

TPC-H~\cite{ref:tpch} is a decision support benchmark that consists of a suite of business oriented ad-hoc queries and concurrent data modifications. The queries and the data populating the database have been chosen to have broad industry-wide relevance while maintaining a sufficient degree of ease of implementation. It  evaluates the performance of various decision support systems by the execution of sets of queries against a standard database under controlled conditions.

The purpose of this benchmark is to reduce the diversity of operations found in an information analysis application, also retaining the application's essential performance characteristics, i.e, the level of system utilization and the complexity  of  operations.  A  large  number  of  queries  of  various  types  and  complexities  need  to  be  executed  to completely  manage  a  business  analysis  environment.

The  components  of  the  TPC-H  database  consists  of  eight  separate  and  individual  tables known as the Base Tables. The relationships between these tables are illustrated in Figure~\ref{fig:tpch}.




\subsection{SSB}
\fig{width=\columnwidth}{ssb}{\textmd{SSB schema}}{fig:ssb}

The Star Schema Benchmark (SSB)~\cite{ref:paper1} is designed to test star schema optimization to address the issues of TPC-H along with measuring performance of database  products  and test  a  new  materialization  strategy. The  SSB  is  a  simple  benchmark  that  consists  of  four  query flights,  four  dimensions,  and  a  simple  roll-up  hierarchy . The SSB is largely based on the TPC-H benchmark with improvements  implememting  a  traditional  pure  star-schema and allowing column and table compression.

The  SSB  is  designed  to  measure the performance  of  database products  against  a  traditional  data  warehouse  scheme.  It  implements  the  same  logical  data  in  a  traditional  star  schema whereas TPC-H models the data in pseudo 3NF schema.

Modifications to  the  TPC-H  schema were made to transform  it  into  a star  schema  form.  The TPC-H tables LINEITEM and ORDERS are combined into one sales fact table named LINEORDER. The PARTSUPP table  is  dropped.  The comment  attributes  for LINEITEMS, ORDERS,  and  shipping instructions  are  also  dropped. A  dimension  table  called DATE is  added  to  the  schema as  is  in  line  with  a  typical  data  warehouse. LINEORDER serves  as the fact  table.  Dimension Tables  are  created  for CUSTOMER, PART, SUPPLIER and DATE.

SSB concentrates on queries that select from the LINEORDER table  exactly  once.  It  avoids   the use  of  self-joins  or  subqueries  as  well  as  or  table  queries also  involving LINEORDER.  The  classic  warehouse  query selects  from  the  table  with  restrictions  on  the  dimension table attributes. SSB supports queries that appear in TPC-H. SSB consists of one large fact table (LINEORDER) and four dimensions tables (CUSTOMER, SUPPLIER, PART and DATE).

%\input{surface}



{
\bibliographystyle{abbrv}
\bibliography{paper}
}
\end{document}
